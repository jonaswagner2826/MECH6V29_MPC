% Standard Article Definition
\documentclass[]{article}

% Page Formatting
\usepackage[margin=1in]{geometry}
\setlength\parindent{0pt}

% Graphics
\usepackage{graphicx}

% Math Packages
\usepackage{physics}
\usepackage{amsmath, amsfonts, amssymb, amsthm}
\usepackage{mathtools}

% Extra Packages
\usepackage{pdfpages}
\usepackage{hyperref}
% \usepackage{listings}

% Section Heading Settings
% \usepackage{enumitem}
% \renewcommand{\theenumi}{\alph{enumi}}
\renewcommand*{\thesection}{Problem \arabic{section}}
\renewcommand*{\thesubsection}{\arabic{section}\alph{subsection})}
\renewcommand*{\thesubsubsection}{}%\quad \quad \roman{subsubsection})}

\newcommand{\Problem}{\subsubsection*{\textbf{PROBLEM:}}}
\newcommand{\Solution}{\subsubsection*{\textbf{SOLUTION:}}}
\newcommand{\Preliminaries}{\subsubsection*{\textbf{PRELIMINARIES:}}}

%Custom Commands
\newcommand{\N}{\mathbb{N}}
\newcommand{\Z}{\mathbb{Z}}
% \newcommand{\Q}{\mathbb{Q}}
\newcommand{\R}{\mathbb{R}}
\newcommand{\C}{\mathbb{C}}

% \newcommand{\SigAlg}{\mathcal{S}}

% \newcommand{\Rel}{\mathcal{R}}

% \newcommand{\toI}{\xrightarrow{\textsf{\tiny I}}}
% \newcommand{\toS}{\xrightarrow{\textsf{\tiny S}}}
% \newcommand{\toB}{\xrightarrow{\textsf{\tiny B}}}

% \newcommand{\divisible}{ \ \vdots \ }
\newcommand{\st}{\ : \ }

% Theorem Definition
% \newtheorem{definition}{Definition}
% \newtheorem{assumption}{Assumption}
% \newtheorem{theorem}{Theorem}
% \newtheorem{lemma}{Lemma}
% \newtheorem{proposition}{Proposition}
% \newtheorem{remark}{Remark}
% \newtheorem{example}{Example}
% \newtheorem{counterExample}{Counter Example}


%opening
\title{
    MECH 6v29 - Model Predictive Control\\ 
    Homework 3
}
\author{Jonas Wagner\\ jonas.wagner@utdallas.edu}
\date{2023, October 20\textsuperscript{th}}

\begin{document}

\maketitle

\tableofcontents

%% Problem 1
\newpage
\section{}
Problem Data:

% 1a)
\subsection{}
Problem Data:
\begin{equation}
    \begin{aligned}
        \mathbf{A} = \mqty[1&1\\0&1] & \mathbf{B} = \mqty[0.5\\1] & \mathcal{W} = \{\vb{w} = \mathbf{B} z \st \abs{z} \leq 0.3\}\\
        \mathbf{C} = \mqty[1&0\\1&0\\0&0] & \mathbf{D} = \mqty[0\\0\\1] & \mathcal{Y} = \{\vb{y} \in \R^3 \st \norm{\vb{y}}_{\infty} \leq 1\}
    \end{aligned}
\end{equation}

Prediction Horizon:
$N = 10$

Initial Condition:
$\vb{x}_0 = 0$

% 1b)
\subsection{Nilpotent candidate controller}
$\Lambda(\mathbf{A+BK}) = 0$

Result w/ Acker:
$\vb{K} = \mqty[-1 & -1.5]$

Same as in \cite{}.


% 1c)
\subsection{Output constraint tightening}
From reference (using different notation):
\begin{equation}
    \begin{aligned}
        \mathcal{Y}_0 = \mathcal{Y}\\
        \mathcal{Y}_{j+1} = \mathcal{Y}_{j} \ominus (\mathbf{C + DK}) \mathbf{L}_{j} \mathcal{W}, \quad \forall_{j \in \{0,\dots,N-1\}}
    \end{aligned}
\end{equation}
where $\mathbf{L}_{j} = (\mathbf{A+BK})^{j}$.\footnote{not explicitly, but eliminating time-variance that's what it is...}

Or equivalently, using the time-invarience of K and some version of the Cayley-Hamilton theorem, 
\begin{equation}
    \mathcal{Y}_{j} = \mathcal{Y} \ominus \bigoplus_{i = 1,\dots,n} (\mathbf{C+DK}) (\mathbf{A+BK})^{i-1}
\end{equation}
(eliminating if the power is negative...)

TODO: double check this... (pretty sure this falls under some distributed property...)

For this system,
$(\mathbf{C+DK}) = \mqty[\vb{I}_2\\\mathbf{K}]$

\begin{equation}
    \begin{aligned}
        \mathcal{Y}_0 &= \{\vb{y} \in \R^3 \st \norm{\vb{y}}_{\infty}\leq 1\}\\
            &= \{\vb{y} \in \R^{3} \st \abs{y_1} \leq 1, \abs{y_2} \leq 1, \abs{y_3}\leq 1\}\\
        \mathcal{Y}_1 &= \mathcal{Y}_0 \ominus (\mathbf{C+DK}) \mathcal{W}\\
            &= \mathcal{Y}_0 \ominus \mqty[\vb{I}_2\\\mathbf{K}] \{ \vb{B} w \in \R \st \abs{w} \leq 0.3\}\\
            &= \{\vb{y} \in \R^{3} \st \abs{y_1} \leq 0.85, \abs{y_2} \leq 0.7, \abs{y_3}\leq 0.4\}\\
        \mathcal{Y}_1 &= \mathcal{Y}_1 \ominus (\mathbf{C+DK}) \mathbf{(A+BK)}\mathcal{W}\\
            &= \mathcal{Y}_1 \ominus \mqty[\vb{I}_2\\\mathbf{K}] \{\vb{B} w \in \R \st \abs{w} \leq 0.3\}\\
            &= \{\vb{y} \in \R^{3} \st \abs{y_1} \leq 0.7, \abs{y_2} \leq 0.4, \abs{y_3}\leq 0.1\}\\
    \end{aligned}
\end{equation}
which is the same for the remaining since $(\mathbf{A+BK})^2 = \vb{0}$.







% \begin{figure}[h]
%     \centering
%     \includegraphics[width = 0.5\textwidth]{figs/pblm1a.png}
%     \caption{Problem 1a results}
% \end{figure}

% The transient behavior is pretty reasonable and has a mild overshoot.
% The maximum output is around 2.5 (2.409) and it appears to initially reach the origin after 4 timesteps but takes around 7 to settle.

% % 1b)
% \newpage
% \subsection{}
% \begin{figure}[h]
%     \centering
%     \includegraphics[width = 0.5\textwidth]{figs/pblm1b.png}
%     \caption{Problem 1b results}
% \end{figure}

% The controller is unable to stabilize the system.

% Is it possible to tune the cost function matrices? 
% Potentially, although I'm not sure how to explicitly prove this either way for every case that this controller would face.
% I did test multiple versions of the matrices but did not find any that satisfied it.

% % 1c)
% \newpage
% \subsection{}
% The first time step ``solution'' is 0.
% Looking at error code we find that the controller result is actually is infeasible or unbounded.

% After it runs for a couple time steps with $u=0$, it then becomes feasible and has a solution that ultimately does stabilize the system.

% \begin{figure}[h]
%     \centering
%     \includegraphics[width = 0.5\textwidth]{figs/pblm1c.png}
%     \caption{Problem 1c results}
% \end{figure}

% Increasing the time does not appear to solve this issue.
% Specifically, increasing it incrementally up to 10 saw no difference, and then running it at $N=50$ also demonstrated no changes.

% % 1d)
% \newpage
% \subsection{}
% % \begin{figure}[h]
% %     \centering
% %     \includegraphics[width = 0.5\textwidth]{figs/pblm1d.png}
% %     \caption{Problem 1d initial results (just w/ a slack variable)}
% % \end{figure}

% I ran this for many different time-horrizons (and the slack variable was weighted with a gain of 100).
% A few of these are included here (the rest available on github).
% Essential takeaway is that the extension of time-horizon allows for a less extreme peak output value and longer period that the output constraint is maintained.
% Additionally, the specific note of combining the two is that feasibility becomes a much larger concern but at a minimum, the constraint at $x_N = 0$ ensures stability (although a slack variable should also be added to ensure feasibility).

% \begin{figure}[h]
%     \centering
%     \includegraphics[width=0.3\textwidth]{figs/pblm1d_N=2.png}
%     \includegraphics[width=0.3\textwidth]{figs/pblm1d_N=4.png}
%     \includegraphics[width=0.3\textwidth]{figs/pblm1d_N=6.png}
%     \includegraphics[width=0.3\textwidth]{figs/pblm1d_N=8.png}
%     \includegraphics[width=0.3\textwidth]{figs/pblm1d_N=10.png}
%     \includegraphics[width=0.3\textwidth]{figs/pblm1d_N=15.png}
% \end{figure}

% \newpage

% %% Problem 2
% \newpage
% \section{}
% \[
%     A = \mqty[2&1\\0&2] \quad B = \mqty[1&0\\0&1]
% \]\[
%     \mathcal{X} = \{x \in \R^2 \st \mqty[1&0] x \leq 5\}
% \]\[
%     \mathcal{U} = \{u \in \R^2 \st -\vb{1} \leq u \leq \vb{1}\}
% \]
% Note: $\mathcal{X} = \{x \in \R^2 \st x_1 \leq 5\}$ 
% and $\mathcal{U} = \{u \in \R^2 \st \norm{u}_{\infty} \leq 1\}$

% Or (in H-rep) $\mathcal{X} = \{\mqty[1&0], 5\}$ 
% and $\mathcal{U} = \{\mqty[\vb{I}_2\\-\vb{I}_2], \mqty[\vb{1}_2\\\vb{1}_2]\}$

% % 2a)
% \subsection{}
% Running it for many different values of alpha resulted in the following:

% \begin{figure}[h]
%     \centering
%     \includegraphics[width=0.3\textwidth]{figs/pblm2a_alpha=1e-01.png}
%     \includegraphics[width=0.3\textwidth]{figs/pblm2a_alpha=5e-01.png}
%     \includegraphics[width=0.3\textwidth]{figs/pblm2a_alpha=1e+00.png}
%     \includegraphics[width=0.3\textwidth]{figs/pblm2a_alpha=2e+00.png}
%     \includegraphics[width=0.3\textwidth]{figs/pblm2a_alpha=1e+01.png}
% \end{figure}
% When $\alpha = 0.1$ the system is unstable.

% % 2b)
% \newpage
% \subsection{}
% When implementing the state and input constraints, the closed loop system is still unstable.
% \begin{figure}[h]
%     \centering
%     \includegraphics[width=0.7\textwidth]{figs/pblm2b}
% \end{figure}
% Additionally, the results become infeasible once it diverges from the origin too much.

% % 2c)
% \newpage
% \subsection{}
% Implementing it with the $x_N = 0$ constraint allows it to be stable (and feasibility issues disappear).
% \begin{figure}[h]
%     \centering
%     \includegraphics[width=0.7\textwidth]{figs/pblm2c}
% \end{figure}

% % 2d)
% \newpage
% \subsection{}
% I had a lot of difficulty getting the batch method to work correctly (see matlab)...
% The specific results requested are here:
% \begin{figure}[h]
%     \centering
%     \includegraphics[width=0.7\textwidth]{figs/pblm2d.png}
% \end{figure}

% The recursive approach was faster as expected (0.16s vs 0.38s).


\newpage
\appendix
\section{Code}
\subsection{Github}
See my github repo for all my course related materials: 
\url{https://github.com/jonaswagner2826/MECH6v29_MPC}

\subsection{Matlab results}
MATLAB code and results are attached.
% \includepdf[pages=-]{HW2.pdf}
% \includepdf[pages=-]{html/HW2_pblm1.pdf}
% \includepdf[pages=-]{html/HW2_pblm2.pdf}


\end{document}
